% HW Template for CS 6150, taken from https://www.cs.cmu.edu/~ckingsf/class/02-714/hw-template.tex
%
% You don't need to use LaTeX or this template, but you must turn your homework in as
% a typeset PDF somehow.
%
% How to use:
%    1. Update your information in section "A" below
%    2. Write your answers in section "B" below. Precede answers for all 
%       parts of a question with the command "\question{n}{desc}" where n is
%       the question number and "desc" is a short, one-line description of 
%       the problem. There is no need to restate the problem.
%    3. If a question has multiple parts, precede the answer to part x with the
%       command "\part{x}".
%    4. If a problem asks you to design an algorithm, use the commands
%       \algorithm, \correctness, \runtime to precede your discussion of the 
%       description of the algorithm, its correctness, and its running time, respectively.
%    5. You can include graphics by using the command \includegraphics{FILENAME}
%
\documentclass[11pt]{article}
\usepackage{tikz}
\usetikzlibrary{positioning}
\newdimen\nodeDist
\nodeDist=20mm
\usepackage{amsmath,amssymb,amsthm}
\usepackage{graphicx}
\usepackage[margin=1in]{geometry}
\usepackage{fancyhdr}
\setlength{\parindent}{0pt}
\setlength{\parskip}{5pt plus 1pt}
\setlength{\headheight}{13.6pt}
\newcommand\question[2]{\vspace{.25in}\hrule\textbf{#1: #2}\vspace{.5em}\hrule\vspace{.10in}}
\renewcommand\part[1]{\vspace{.10in}\textbf{(#1)}}
\newcommand\approach{\vspace{.10in}\textbf{Approach: }}
\newcommand\algorithm{\vspace{.10in}\textbf{Algorithm: }}
\newcommand\assumption{\vspace{.10in}\textbf{Assumption: }}
\newcommand\correctness{\vspace{.10in}\textbf{Correctness: }}
\newcommand\runtime{\vspace{.10in}\textbf{Running time: }}
\pagestyle{fancyplain}
\lhead{\textbf{\NAME\ (\UID)}}
\chead{\textbf{HW\HWNUM}}
\rhead{CS 6150, \today}
\begin{document}\raggedright
%Section A==============Change the values below to match your information==================
\newcommand\NAME{Aishwarya Asesh}  % your name
\newcommand\UID{u1063384}     % your utah UID
\newcommand\HWNUM{3}              % the homework number
%Section B==============Put your answers to the questions below here=======================

% no need to restate the problem --- the graders know which problem is which,
% but replacing "The First Problem" with a short phrase will help you remember
% which problem this is when you read over your homeworks to study.


\question{1}{Easy relatives of 3-SAT}
\part{a}
Truth table for the following problem may be written as:\\
\bgroup 
\def\arraystretch{1.2}
\begin{tabular}{|l|c|c|c|c|} \hline 
{\bf \underline {$x_1$}} & {\bf \underline {$x_2$}} & {\bf \underline {$ \neg x_1$}} & {\bf \underline {$x_{1} \lor x_{2}$}} & {\bf \underline {$\neg x_{1} \implies x_{2}$}} \\ \hline

0 & 0 & 1 & 0 & 0 \\ \hline
0 & 1 & 1 & 1 & 1 \\ \hline
1 & 0 & 0 & 1 & 1 \\ \hline
1 & 1 & 0 & 1 & 1 \\ \hline

\end{tabular}
\egroup
\\As we can observe in the truth table above, the output values of  {$x_{1} \lor x_{2}$} and {$\neg x_{1} \implies x_{2}$} are same for respective values of $x_1$ and $x_2$, Hence both the expressions are logically same.\\
\part{b}
Let us plot a graph between literals.\\
\assumption We can assume that there exist a link from $x_i$ to $\neg x_i$ and from $\neg x_i$ to $x_i$. Let us assume $p$ to be a satisfying assignment value. So we can assume $p(x)$ = TRUE.\\
A new edge will be made in the graph if there is a clause $A \implies B$ or $\neg A \lor B$. This implies that if A is TRUE, B must be TRUE for the clause to be TRUE. If there is an connected edge between $x_i$ to $\neg x_i$ and from $\neg x_i$ to $x_i$, the clause $A \implies B$ or $\neg A \lor B$ will give FALSE value for $x_i$ equal to TRUE.\\
But this statement contradicts our assumption and hence we can say that it is not possible to have an edge connection between $\neg x_i$ and $x_i$. Similar case exists for $p(x)$ = FALSE.\\
Thus we can say that 2-CNF formula is satisfiable if and only if there is no variable $x_i$ for which there is a
path from $x_i$ to $\neg x_i$ and vice versa.\\
\runtime Using Breadth-first search or Depth-first search we can find if $x_i$ and $\neg x_i$ has an edge connection between them. This can be done in polynomial time. If an edge is found between them, we can discard as it wont be satisfiable.\\
Thus, 2-SAT has a polynomial time algorithm.

\part{c}
We will assume a 3 - SAT instance as F = $(x_{1} \lor x_{2} \lor x_{3})$ $\land$ $(x_{4} \lor x_{5} \lor x_{6})$. As we can observe in the first clause, $(x_{1} \lor x_{2} \lor x_{3})$, according to the requirements, atleast 2 literals need to be TRUE to satisfy the instance. Thus we can say that the F instance is satisfied if and only if one literal in $(x_{1} \lor x_{2})$, $(x_{2} \lor x_{3})$, $(x_{3} \lor x_{1})$ is satisfied. For further clarification we can consider if $x_1$ is TRUE, $(x_{1} \lor x_{2})$, $(x_{3} \lor x_{1})$ 
is satisfied, and for one literal in $(x_{2} \lor x_{3})$ must be TRUE, $x_2$ or $x_3$ must be TRUE.\\
Thus we can see a similar case to 2 - SAT problem, atleast one literal in $(x_{1} \lor x_{2})$, $(x_{2} \lor x_{3})$, $(x_{3} \lor x_{1})$ should be TRUE. We know a polynomial time algorithm exist for a 2 - SAT problem. Hence we can conclude that 2-or-more 3 - SAT is in Polynomial time.
 
\question{2}{Decision vs Search}
We can write the following pseudo code to show the explaination as required:\\
$Find\_Independent\_Set(G,k)$:\\
1. If FALSE is returned by Oracle O or else if k is 0, Stop and return FALSE.\\
2. Let $v$ $\epsilon$ $G(V)$.\\
3. Let $Neigh(v)$ be the adjacent node of $v$.\\
4. We can further create 2 $subgraphs$, $SG_1$ and $SG_2$ with $G - v$ \& $G - v - Neigh(v)$ respectively. Graph $SG_2$ is made as to make a graph such that if adjacent vertex are present for a particular vertex, it wont be having any independent set in it.\\
5. Now we can give $SG_2$ as input to Oracle O. If the given input value returns TRUE:\\
return $v$ $\cup$ $Find\_Independent\_Set(SG_2,k-1)$. Here as we see that one vertex is found, so we start recursing on $k-1$.\\
6. Finally return $Find\_Independent\_Set(G,k)$   

\question{3}{Reductions, Reductions}
 
\part{a}
Let's F be an instance of SAT, where\\
F = $(x_{1} \lor x_{2} \lor x_{3} \lor x_{4})$ $\land$ $(x_{5} \lor x_{6} \lor x_{7})$ $\land$ $(x_{8} \lor x_{9} \lor x_{10})$\\
As we know, for satifiablity of SAT problem, each clause should have atleast one literal as TRUE.\\
For example: $(x_{1} \lor x_{2} \lor x_{3} \lor x_{4})$ must have atleast one literal as TRUE, which implies $(x_{1} \lor x_{2} \lor x_{3} \lor x_{4})$ is greater than or equal to one. Similarly, $(x_{5} \lor x_{6} \lor x_{7})$ must have atleast one literal as TRUE, which implies $(x_{5} \lor x_{6} \lor x_{7})$ is greater than or equal to one and so on.\\
This can be considered as an instance of Integer Linear Programming problem. The same proof can be extended for every clause in SAT instance, and thus Integer Linear Programming can be considered to be as hard as SAT problem. 

\part{b}
To Prove: Integer Linear Programming is an NP Complete Problem\\
As we see in the above problem - part (a), Integer Linear Programming is an NP hard problem. Thus if we prove that an instance of Integer Linear Programming can be verified in polynomial time, by giving an assignment, it can be proved that ILP is NP Complete. The above can be done by putting the corresponding assignment values into the instance comparing the inequality. This can be achieved in polynomial time. Thus, Integer Linear Programming is NP complete.

\part{c}
LINEQ(modulo2) has a poly time algo.
\\ Consider variable in 3-SAT be $z_1,z_2,z_3....z_n$. Thus we will have respective variables $y_1,y_2,y_3....y_n$ in ILP. We restrict each variable to be 0/1:
\\ $0<= y_i<= 1$ for all values to i.
\\ Initializing $y_i$=1 in ILP and defining $z_i$ as TRUE in the given formula, similar case for $y_i$=0 shows $z_i$=FALSE.
\\ For every clause i.e. $(z_{1} \lor \neg z_{2} \lor z_{3})$ will have limitations as:
\\$y_1 + (1-y_2) + y_3$ $>$ 0
\\ To properly satisfy the above equation we should set $y_1=1$ or $y_2=0$ or $y_3=1$ , which implies we must set $z_1$=TRUE or $z_2$=FALSE or $z_3$=TRUE in respective TRUTH assignments.  


\part{d}
{\bf Input} Let there be f quadratic equations having boolean literals $x_1,x_2,....,x_n$.
\\ Here we can say that every equation is in following form:
\\ $\Sigma_{i,j \epsilon [n]} x_{i}x_{j} = b$, where addition is for modulo 2.
\\ if there exist a 0/1 assignment of variables that satisfy every equation then it is solvable.
\\ {\bf To Show} Quadeq is NP Complete
\\ {\bf Proof} If we find the decision problem belongs to NP - class for values of $x_1,x_2,....,x_n$, we can verify that each equation is satisfied in poly time.
\\ Lets proceed by reduction of 3- SAT to Quadratic Equation.
\\ {\bf Reduction}: We can reduce every clause in 3-SAT instance in an similar Quadratic Equation in poly time using the following rules:
\\ 1. If a particular literal like $x_i$ is in non-negative state then we can reduce it to simply $k_i$ and if it exists in $\neg x_i$ state then we can reduce the literal to $(1-k_i)$.
\\ 2. Multiply the reduced term with any variable such as $p$. Hence we can say that if $x_i$ is TRUE in the clause, variable $k_i$ will be having value 1 and vice-versa if $x_i$ is FALSE then $k_i$ will have a ZERO value.
\\ 3. Example: if any clause in 3-SAT instance is in the state $x_1 \lor \neg x_2 \lor x_3$ then its reduced form of Quadratic equation will be $p k_1 + q (1 - k_2) + r k_3$ $=$ 1 modulo 2. Here p,q,r are boolean variables that can be modified to satisfy the requirement functions.
\\ 4. Hence, with respect to f clauses in 3-SAT, there will be f Quadratic Equations present.
\\ {\bf Claim 1:} If we consider 3-SAT instance as satisfiable then all Quadratic Equation will be satisfied.
\\ We know that one literal in a 3-SAT must be TRUE, if a clause in 3-SAT instance is considered TRUE.  Thus , this literal will give value 1 to the reduced state. The boolean variable attached with it will have value 1, rest all boolean variables can have value 0. Thus this equation will get equated to value 1. Same case is TRUE for satisfiability of all quadratic equation.
\\ {\bf Claim 2:} If we consider all Quadratic Equations as satisfied then 3-SAT instance is satisfiable.
\\ We can observe that if a particular Quadratic Equation is equated to value 1, then atleast one reduced variable must have the value such that the respective literal in 3-SAT clause is TRUE, since Quadratic Equation can equate to one by changing the boolean variables, except when all terms multiplied are 0. Hence if all Quadratic Equations are satisfied then we can say that all the clauses will evaluate to TRUE.
\\ Hence using claim 1 and 2 we can say that Quadratic Equations is NP Hard, and since it also belongs to NP- class, it must be NP complete. 
 
\question{4}{Graphs - definitions}
\part{a}
We can prove the required statement using contradiction.\\
\assumption Let G be a finite graph, that has no cycle. We assume that the graph G has longest path ending at vertex $v_e$\\
It is stated that Graph has degree greater than or equal to 2. As we have considered, since $v_e$ is the last vertex of the longest path in Graph G, it must have a incoming edge. But according to statement "Graph has degree greater than or equal to 2" the other edge must join with some other vertex in the Graph, thus forming a Cycle. In other case vertex $v_e$ will have to extend the longest path, which is contradicting the statement that Graph G has no cycle.\\
Thus Graph G must have a cycle if every vertex has degree greater than or equal to 2.
  
\part{b}
As we know, in a Graph, the sum of degree of vertices is twice the number of edges.\\
Sum of degree of vertices in given graph is 35.\\
Thus accordingly the number of edges should be 17.5, that is not feasible.\\
Thus we can say that such a Graph is not possible.

\part{c}
We can solve the following problem using Depth - first search. The previous or back edges can be detected. Counting of number of elements in stack happens when an edge is found.\\
1. Let a Graph be G. Mark all the nodes or vertices as 0 or unvisited.\\
2. Create an array N, with False or Null value for every vertex in Graph G.\\
3. Use depth - first search for every vertex $v$ in Graph G.\\
4. If for a particular node, visited is FALSE or 0, then:\\
i. Mark this as 1 or visited, and change N for this node as TRUE.\\
ii. Now for next step, recurse on the neighboring vertices for this particular vertex. In case n is 1 or TRUE, for a vertex, we know that there must be a existing cycle.\\
iii. Next step is to find number or 1 or TRUE cases in N, if it is odd return all vertices that contains the 1 or TRUE value, else return FALSE, as there does not exist odd number of vertices.\\

\question{5}{Weary traveler}
We can formulate the whole problem statement as:\\
Let:\\
Vertices be airports\\
Edges be flights with weights depart timing from origin airport as Depart\_T and arrive timing at destination Airport be Arrive\_T\\
We consider Time as a dictionary with key as node and value as the Arrival\_T\\
Thus T[A] = Arrival\_T at A start time.\\

Airport\_Problem (Graph,A,InitialTime)\\
1. Consider T[A] with start time for the travel. Initially T[v] for all vertices that exist in the graph is considered as $\infty$\\
2. Priority Queue PQ exists of vertices on T. Here low T means high priority as main aim is to reduce time taken.\\
3. If PQ is not NULL, select a vertex v having the highest priority value.\\
4. For every vertex u in the adjacency list of v, create another Priority Queue, F for flights with Depart\_T minus Arrive\_T is greater than equal to 10 minutes, on Arrival\_T where low Arrival time means higher priority. Thus in this way we can give priority to flights that arrive earliest.\\
5. Now, we find the flight with highest priority and let PT be its arrival time.\\
6. If for condition PT is less than equal to T[u], update T[u] with the values of ReturnT[B].\\

\runtime This is a modified version of Dijkstra's algorithm with Priority Queue as an additional feature.\\
Thus Dijkstra's algorithm takes $O(n^2)$ time\\
Creating priority queue will take $O(m)$ time in the worst case that is the number of flights.\\
So the total cost is $O(n^2)$ + $O(m)$ = $O(u^2)$. 
 
\end{document}